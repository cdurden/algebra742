\documentclass[11pt,article,landscape]{memoir}\usepackage[]{graphicx}\usepackage[]{color}
% maxwidth is the original width if it is less than linewidth
% otherwise use linewidth (to make sure the graphics do not exceed the margin)
\makeatletter
\def\maxwidth{ %
  \ifdim\Gin@nat@width>\linewidth
    \linewidth
  \else
    \Gin@nat@width
  \fi
}
\makeatother

\definecolor{fgcolor}{rgb}{0.345, 0.345, 0.345}
\newcommand{\hlnum}[1]{\textcolor[rgb]{0.686,0.059,0.569}{#1}}%
\newcommand{\hlstr}[1]{\textcolor[rgb]{0.192,0.494,0.8}{#1}}%
\newcommand{\hlcom}[1]{\textcolor[rgb]{0.678,0.584,0.686}{\textit{#1}}}%
\newcommand{\hlopt}[1]{\textcolor[rgb]{0,0,0}{#1}}%
\newcommand{\hlstd}[1]{\textcolor[rgb]{0.345,0.345,0.345}{#1}}%
\newcommand{\hlkwa}[1]{\textcolor[rgb]{0.161,0.373,0.58}{\textbf{#1}}}%
\newcommand{\hlkwb}[1]{\textcolor[rgb]{0.69,0.353,0.396}{#1}}%
\newcommand{\hlkwc}[1]{\textcolor[rgb]{0.333,0.667,0.333}{#1}}%
\newcommand{\hlkwd}[1]{\textcolor[rgb]{0.737,0.353,0.396}{\textbf{#1}}}%
\let\hlipl\hlkwb

\usepackage{framed}
\makeatletter
\newenvironment{kframe}{%
 \def\at@end@of@kframe{}%
 \ifinner\ifhmode%
  \def\at@end@of@kframe{\end{minipage}}%
  \begin{minipage}{\columnwidth}%
 \fi\fi%
 \def\FrameCommand##1{\hskip\@totalleftmargin \hskip-\fboxsep
 \colorbox{shadecolor}{##1}\hskip-\fboxsep
     % There is no \\@totalrightmargin, so:
     \hskip-\linewidth \hskip-\@totalleftmargin \hskip\columnwidth}%
 \MakeFramed {\advance\hsize-\width
   \@totalleftmargin\z@ \linewidth\hsize
   \@setminipage}}%
 {\par\unskip\endMakeFramed%
 \at@end@of@kframe}
\makeatother

\definecolor{shadecolor}{rgb}{.97, .97, .97}
\definecolor{messagecolor}{rgb}{0, 0, 0}
\definecolor{warningcolor}{rgb}{1, 0, 1}
\definecolor{errorcolor}{rgb}{1, 0, 0}
\newenvironment{knitrout}{}{} % an empty environment to be redefined in TeX

\usepackage{alltt}
% Copyright (C) 2013 Andrew Gainer-Dewar <andrew.gainer.dewar@gmail.com>
% This file may be distributed and/or modified under the
% conditions of the LaTeX Project Public License, either
% version 1.2 of this license or (at your option) any later
% version. The latest version of this license is in:
% http://www.latex-project.org/lppl.txt
% and version 1.2 or later is part of all distributions of[
% LaTeX version 1999/12/01 or later.

\usepackage{fullpage}
\usepackage{agd-rubric}
\usepackage{tikz}

\rubriccourse{Algebra 1.1\\
Vocabulary: 4/10, Analysis: 1/3, Correct Solutions to Questions: 1, 2, 3, 4, 5, 6, 7, 8, 10, 11, 12 
Final score: \textbf{62/100}
}
%\rubricterm{}
\rubricthing{Solving Equations Test}
\rubrictopprompt{Student}
\rubrictoppromptfill{Monika Sanchez Mercado\hrulefill}

\def\checkmark{\tikz\fill[scale=0.8](0,.35) -- (.25,0) -- (1,.7) -- (.25,.15) -- cycle;}
\IfFileExists{upquote.sty}{\usepackage{upquote}}{}
\begin{document}
\maketitle

% The argument determines the number of score buckets
\begin{rubrictable}{4}
  % Add your \rubricdesc items in increasing order!
  \rubriccat{Understanding mathematical vocabulary}{
  Score: 38/100
}{
    \rubricdesc{Not demonstrate}{
    \checkmark
    }
    \rubricdesc{I can identify examples of some key concepts.}{
    
    }
    \rubricdesc{I can identify the examples of most key concepts.}{
    
    }
    \rubricdesc{I can identify the examples of all key concepts and I can use the definitions of key concepts accurately to answer questions.}{
    
    }
  }
  \rubriccat{Solving equations}{
  Score: 93/100
}{
    \rubricdesc{Not demonstrate}{
    
    }
    \rubricdesc{I can solve equations where a variable appears one time in a sum or product with both a constant and a coefficient.}{
    \checkmark
    }
    \rubricdesc{I can solve equations where the variable appears in multiple places on the same side of the equals sign.}{
    \checkmark
    }
    \rubricdesc{I can solve equations where the variable appears on both sides of the equals sign. I can solve equations involving absolute value.}{ 
    \checkmark
    }
  }
  \rubriccat{Communicating my reasoning}{
  Score: 25/100
}{
    \rubricdesc{Not demonstrate}{
    \checkmark
    }
    \rubricdesc{I present my reasoning for some problems. }{
    
    }
    \rubricdesc{I present my reasoning for every problem. }{
    
    }
    \rubricdesc{I can present a detailed argument, including all necessary steps needed to reach all of my conclusions.}{
    
    }
}
\end{rubrictable}
\end{document}
